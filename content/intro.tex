\section{Introduction}

\begin{frame}{Some Physics}

    \begin{block}{Classical Mechanics}
        \begin{itemize}
            \item In classical mechanics, the space of all possible states of a system is given by \emph{phase space}, $M$.
            \item ``\emph{State}'' describes the \emph{position} and the \emph{momentum}.
        \end{itemize}
    \end{block}

    \begin{block}{Quantum Mechanics}
        \begin{itemize}
            \item In quantum mechanics, still have phase space $M$ but states are replaced by \emph{wavefunctions}.
            \item In my research, wavefunctions are just \emph{homogeneous polynomials}, for example:
            $$ \phi(\vb{z}) = z_{1}^{k_{1}}z_{2}^{k_{2}}z_{3}^{k_{3}}, \qquad \text{where } k_{1} + k_{2} + k_{3} = k. $$
        \end{itemize}
    \end{block}

\end{frame}