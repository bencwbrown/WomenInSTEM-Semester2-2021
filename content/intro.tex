\section{Introduction}

\begin{frame}{Classical Mechanics}

    \begin{itemize}
        \item In classical mechanics, the space of all possible states of a system is given by \emph{phase space}, $M$.
        \item ``\emph{State}'' means not just the \emph{position} but also the \emph{momentum} (of a particle).
    \end{itemize}
    
    \begin{block}{Example}
        For a particle of unit mass, energy function $E : M \ra \RR$:
        \begin{equation*}
            E(x,p) = \frac{p^{2}}{2} + V(x).
        \end{equation*}
    \end{block}

    \begin{equation*}
        \rightsquigarrow \frac{\partial E}{\partial x} = \frac{\partial V}{\partial x} = -\frac{\partial p}{\partial t}, \quad \text{and} \quad \frac{\partial E}{\partial p} = p = \frac{\partial x}{\partial t}.
    \end{equation*}

\end{frame}

\begin{frame}{Quantum Mechanics}

    \begin{itemize}
        \item Still have phase space $M$ and energy $E$, but replaces states with \emph{wavefunctions}.
        \item Classical energy function $E : M \ra \RR$ now becomes an operator $\hat{E} : M \ra M$.
    \end{itemize}

    \begin{block}{Question}
        For a phase space $M$, what is $\mc{Q}(M)$?

    \begin{itemize}
        \item Phase space $M$ represents the position and momentum of a system, with energy function $E : M \ra \RR$.
        \item Quantisation $\mc{Q}(M)$ should then represent the wavefunctions $\psi : M \ra \CC$, with operators $\hat{E} : M \ra M$.
    \end{itemize}
    \end{block}

\end{frame}